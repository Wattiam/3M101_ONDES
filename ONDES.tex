\documentclass[10pt,a4paper]{article}
\usepackage[utf8]{inputenc}
\usepackage[francais]{babel}
\usepackage[T1]{fontenc}
\usepackage{amsmath}
\usepackage{amsfonts}
\usepackage{amssymb}
\usepackage{makeidx}
\usepackage{graphicx}
%\usepackage{lmodern}
%\usepackage{fourier}
\usepackage[left=2cm,right=2cm,top=2cm,bottom=2cm]{geometry}
\usepackage{titlepic}
\usepackage{pythontex}
\usepackage{lipsum}
\usepackage{fancyhdr}
\author{BELHADJI, Yanis ; CAZALS, Laure ; GUNNY Masoodah ; RODRÍGUEZ CORTÉZ, Carlos Alfonso}
\title{}






\begin{document}
 %\thispagestyle{empty}
\begin{titlepage}
\includegraphics[scale=0.4]{logo.png}
\begin{center}
\vspace*{7cm}
{\Huge  \textbf{UE 3M101 - }}{\huge \textbf{PROJET}}
\\

 \vspace{1.5cm}
 \huge \textbf{ONDES}
 \vfill

{\large  Encadrant : CAZENAVE, Thierry}
\\~\\
{\normalsize BELHADJI, Yanis ; CAZALS, Laure ; GUNNY Masoodah ; RODRÍGUEZ CORTÉZ, Carlos Alfonso}
\\
\vspace{1cm}
{\normalsize Année 2019-2020}
 \end{center}

\end{titlepage}
 
 
\thispagestyle{empty}
\vspace*{\fill}
\renewcommand{\contentsname}{\Huge Sommaire}
\tableofcontents
\vspace{\fill}


\pagebreak
\setcounter{page}{1}

\section{Introduction}
La première équation aux dérivées partielles est apparut pour expliquer le phénomène de vibration d'une corde fixée à ses extremités, et fut présentée sous la forme suivante par Jean le Rond d'Alembert :
\begin{equation} \label{Ondes} 
\begin{cases} 
\displaystyle \frac {\partial ^2 u } {\partial t ^2} = \frac {\partial ^2 u } {\partial x ^2}  \quad  \text{pour}\quad t\ge 0, x\in (0,\pi ) \\ u(t, 0)= u (t, \pi )= 0\quad  \text{pour} \quad t\ge 0
\end{cases} 
\end{equation} 
où la fonction $u(t, x)$ représente le déplacement transversal du point $x$ de la corde à l'instant $t$ par rapport à sa position d'équilibre (qui est $u_{eq} = 0$ lorsqu'on considère que la corde est tendue). Cette équation correspond au cas où les oscillations $\mid u\mid$ sont suffisament petites.
D'autres équations et systèmes d'équations différentielles ont été proposés pour mieux décrire les phénomènes vibratoires en physique. Ces objets sont d'une grande importance par exemple en mécanique des fluides, en optique, voire en mécanique quantique.\\
L'objectif de ce projet est de présenter quelques cas d'équations différentielles décrivant des ondes, d'en dériver les solutions au moyen d'un calcul numérique et d'en discuter les particularités, notamment les différents régimes d'oscillation auxquels donnent naissance ces différents systèmes.
On étudiera les systèmes :
\begin{equation} \label{KC} 
\begin{cases} 
\displaystyle \frac {\partial ^2 u } {\partial t ^2} =  \Bigl( 1+ \int _0^\pi  \Bigl( \frac {\partial u} {\partial x} \Bigr)^2 \Bigr) \frac {\partial ^2 u } {\partial x ^2} \quad  \text{pour}\quad t\ge 0, x\in (0,\pi ) \\ u(t, 0)= u (t, \pi )= 0\quad  \text{pour} \quad t\ge 0
\end{cases} 
\end{equation} 



\begin{equation} \label{CHW} 
\begin{cases} 
\displaystyle \frac {\partial ^2 u } {\partial t ^2} = \frac {\partial ^2 u } {\partial x ^2} -  \Bigl( \int _0^\pi u(t,s)^2 ds \Bigr) u \quad  \text{pour}\quad t\ge 0, x\in (0,\pi ) \\ u(t, 0)= u (t, \pi )= 0\quad  \text{pour} \quad t\ge 0
\end{cases} 
\end{equation} 

Le système \eqref{KC} est l'équation de Kirchhoff (ou Kirchhoff-Carrier), et le système \eqref{CHW} est un modèle plus simple qui partage certaines propriétés fondamentales avec \eqref{KC}.


\section{Cordes vibrantes et intéractions en modes simples}

\section{Étude du système \eqref{CHW}}

    \subsection{Simplification de l'équation}
    \subsection{Résolution numérique}
    \subsection{Phénomène de transfert d'énergie (solutions exceptionnelles)}
\section{Résolution de l'équation de Kirchhoff (système \eqref{KC})}
    \subsection{Simplification de l'équation}
    \subsection{Résolution numérique}
    \subsection{Phénomène de transfert d'énergie (solutions exceptionnelles) (peut-être??)}

\section{Solution à un système inédit}

\end{document}



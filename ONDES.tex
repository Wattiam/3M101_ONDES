\documentclass[10pt,a4paper]{article}
\usepackage[utf8]{inputenc}
\usepackage[francais]{babel}
\usepackage[T1]{fontenc}
\usepackage{amsmath}
\usepackage{amsfonts}
\usepackage{amssymb}
\usepackage{makeidx}
\usepackage{graphicx}
\usepackage{lmodern}
\usepackage{fourier}
\usepackage[left=2cm,right=2cm,top=2cm,bottom=2cm]{geometry}
\usepackage{titlepic}
\author{BELHADJI, Yanis ; CAZALS, Laure ; GUNNY Masoodah ; RODRÍGUEZ CORTÉZ, Carlos Alfonso}
\title{%
Projet 3M101 Ondes\\ 
\large Encadré par : Thierry Cazenave\\
Sorbonne Université\\
\small année 2019-2020}
\begin{document}
\maketitle
\pagebreak
\tableofcontents
\pagebreak
\section{Introduction}
La première équation aux dérivées partielles apparut pour expliquer le problème de la vibration d'une corde fixée à ses extremités, et fut présentée sous la forme suivante par Jean le Rond d'Alembert :
\begin{equation} \label{Ondes} 
\begin{cases} 
\displaystyle \frac {\partial ^2 u } {\partial t ^2} = \frac {\partial ^2 u } {\partial x ^2}  \quad  \text{pour}\quad t\ge 0, x\in (0,\pi ) \\ u(t, 0)= u (t, \pi )= 0\quad  \text{pour} \quad t\ge 0
\end{cases} 
\end{equation} 
où la fonction $u(x, t)$ représente le déplacement transversal du point $x$ de la corde à l'instant $t$. Cette équation correspond au cas où les oscillations $\mid u\mid$ sont suffisament petites.
D'autres équations et systèmes d'équations différentielles ont été proposés pour mieux décrire les phénomènes vibratoires en physique. Ces objets sont d'une grande importance par exemple en thermodynamique, en optique, voire en mécanique quantique.\\
L'objectif de ce projet est de présenter quelques cas d'équations différentielles décrivant des ondes, d'en dériver les solutions au moyen d'un calcul numérique et d'en discuter les particularités, notamment les différents régimes d'oscillation auxquels donnent naissance ces différents systèmes.

\section{Cordes vibrantes et intéractions en modes simples}

\section{Étude du premier système}
\subsection{Simplification de l'équation}
\subsection{Résolution numérique}
\subsection{Phénomène de transfert d'énergie (singularité)}
\section{Résolution de l'équation de Kirchhoff}
\section{Solution à un système inédit}

\end{document}\documentclass[10pt,a4paper]{article}
\usepackage[utf8]{inputenc}
\usepackage[francais]{babel}
\usepackage[T1]{fontenc}
\usepackage{amsmath}
\usepackage{amsfonts}
\usepackage{amssymb}
\usepackage{makeidx}
\usepackage{graphicx}
\usepackage{lmodern}
\usepackage{fourier}
\usepackage[left=2cm,right=2cm,top=2cm,bottom=2cm]{geometry}
\usepackage{titlepic}
\author{BELHADJI, Yanis ; CAZALS, Laure ; GUNNY Masoodah ; RODRÍGUEZ CORTÉZ, Carlos Alfonso}
\title{%
Projet 3M101 Ondes\\ 
\large Encadré par : Thierry Cazenave\\
Sorbonne Université\\
\small année 2019-2020}
\begin{document}
\maketitle
\pagebreak
\tableofcontents
\pagebreak
\section{Introduction}
La première équation aux dérivées partielles apparut pour expliquer le problème de la vibration d'une corde fixée à ses extremités, et fut présentée sous la forme suivante par Jean le Rond d'Alembert :
\begin{equation} \label{Ondes} 
\begin{cases} 
\displaystyle \frac {\partial ^2 u } {\partial t ^2} = \frac {\partial ^2 u } {\partial x ^2}  \quad  \text{pour}\quad t\ge 0, x\in (0,\pi ) \\ u(t, 0)= u (t, \pi )= 0\quad  \text{pour} \quad t\ge 0
\end{cases} 
\end{equation} 
où la fonction $u(x, t)$ représente le déplacement transversal du point $x$ de la corde à l'instant $t$. Cette équation correspond au cas où les oscillations $\mid u\mid$ sont suffisament petites.
D'autres équations et systèmes d'équations différentielles ont été proposés pour mieux décrire les phénomènes vibratoires en physique. Ces objets sont d'une grande importance par exemple en thermodynamique, en optique, voire en mécanique quantique.\\
L'objectif de ce projet est de présenter quelques cas d'équations différentielles décrivant des ondes, d'en dériver les solutions au moyen d'un calcul numérique et d'en discuter les particularités, notamment les différents régimes d'oscillation auxquels donnent naissance ces différents systèmes.

\section{Cordes vibrantes et intéractions en modes simples}

\section{Étude du premier système}
\subsection{Simplification de l'équation}
\subsection{Résolution numérique}
\subsection{Phénomène de transfert d'énergie (singularité)}
\section{Résolution de l'équation de Kirchhoff}
\section{Solution à un système inédit}

\end{document}


